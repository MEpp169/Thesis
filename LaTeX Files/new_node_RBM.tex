%%% How The RBM->UBM transition arises naturally when implementing Gates %%%


\documentclass{article}
\usepackage{amsmath}
\usepackage{amssymb}

\begin{document}
%
%
\section{Unitary Operations and RBMs}
%
The state of an $N$-spin system is described by a wavefunction $\Psi(s)$ with
$s\in\{0,1\}^N$. $\Psi(s)$ can be encoded in a so-called Boltzmann Machine, a
simple two-layer ANN. It consists of $N_v$ visible and $N_h$ hidden nodes and
is specified by parameters $a\in\mathbb{R}^{N_v}, b\in\mathbb{R}^{N_h}$ and
$w\in\mathbb{R}^{N_h\times N_v}$. The energy function
\begin{equation}
    E(v,h) = v^Ta + h^Tb + h^Twv
\end{equation}
allows one to express probability distributions defined on $\mathbb{R}^{N_v}$
according to
\begin{equation}
    P(v) = \frac{\sum_{h}e^{-E(v,h)}}{\sum_v \sum_h e^{-E(v,h)}}.
\end{equation}
Similarly, if we allow complex $a\in\mathbb{C}^{N_v}, b\in\mathbb{C}^{N_h}$ and
$w\in\mathbb{C}^{N_h\times N_v}$ we can map each spin configuration
$s\in\{0,1\}^{N_S}$ of system with $N_S$ spins to a complex amplitude. This is
then the wavefunction
\begin{equation}
    \Psi(s) = \frac{\sum_{h}e^{-E(s,h)}}{\sum_s \sum_h e^{-E(s,h)}}.
    \label{eq:RBM-psi}
\end{equation}
Consider now a 1-body unitary operator $O\in U(2)$. It is completely specified
by its four matrix elements $O_{ss'}$ where $s,s'\in\{0,1\}$. Equivalently, we
can express this Operator in terms of an exponential function:
\begin{equation}
    O(s, s') = A \exp{(\alpha s + \beta s' +  \omega ss')}.
\end{equation}
To do this we simply associate
\begin{align*}
    A &= O_{00} \\
    \alpha &= \ln{\left(\frac{O_{10}}{A}\right)} \\
    \beta &= \ln{\left(\frac{O_{01}}{A}\right)} \\
    \omega &= \ln{\left(\frac{O_{11}}{A}\right)} - (\alpha + \beta).
\end{align*}
Clearly the function $O(s,s')$ defined this way behaves like the operator. The
wavefunction $\Psi'(s)$ after $O$ has acted upon the initial state $Psi(s)$ is
given by
\begin{align}
    \Psi'(s) &= \sum_{s'} O_{ss'} \Psi(s_1,\hdots, s', \hdots, s_N) \nonumber \\
             &= \sum_{s'} \exp{(\alpha s + \beta s' +  \omega ss')} \Psi(s_1,
                \hdots, s', \hdots, s_N)
\end{align}
Expressing the RHS wave-function now in terms of our RBM (cf. \ref{eq:RBM-psi}),
the sum over spin $s'$ can be associated with another hidden node.
%
%
%
\section{Unitary Operations and POVM Distributions}
Suppose now the RBM encodes some probability distribution $P(\vec{a})$ of an
informationally complete POVM measurement. This distribution then uniquely
specifies some quantum state. Let $O_{\vec{a},\vec{b}}$ now denote a single-body
unitary operator acting on POVM distribution vectors. Is it still possible to
rewrite this operator as an exponential function and describe the effect of the
unitary operation as the insertion of an additional node into the RBM? \par



\end{document}
